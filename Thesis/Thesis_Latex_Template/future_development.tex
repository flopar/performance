\chapter{Future Development}
Looking back at the beginning, the project started with small goals, which during the research and development phase, proved to be more complex and have a much higher scalability. Despite this fact the first steps were successfully taken and this showed that the project is not only a dream, but can become reality. With all this said, there is still room for improvement. In time, older tools will change or disappear and newer ones will take their place. Because of that the maintenance will be crucial if this project will live on for more time. At the time of writing I can think only of the following points that will need some improvements:%As mentioned at the beginning, the current state of the library is not one that I'm proud of. This is because of the lack of expertise, research- and development time. The following points should be reworked or implemented in the future:
\begin{enumerate}
	\item CMake compiler support: although CMake recognizes the compiler by itself, there are a lot of cases where teams develop their software for more than one compiler on the same machine. Therefore it would be a convenience to have a CMake file template that searches for compilers on the calling machine and sets the appropriate CMake variables
	\item Windows Fibers support: in my research about threads for different architectures, windows fibers were mentioned. According to Microsoft Docs \dq fibers do not provide advantages over a well-designed multithreaded application. However, using fibers can make it easier to port applications that were designed to schedule their own threads\dq{}\cite{windows-fibers-doc}
	\item Error Handling: as of now, personally, I'm not satisfied by the way some functions handle errors. 
	\item Naming conventions: some methods have similar names, because they do almost the same thing, although in some cases a more creative name would be more pleasing
	\item CPU's affinity: this subject should be researched more and be also implemented for other operating systems
	\item Linking: the current state of the library links its components statically. If possible, the components should be linked dynamically, in order to minimize the size of the calling executable	
\end{enumerate}