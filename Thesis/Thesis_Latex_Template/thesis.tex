\documentclass{wissdoc}
%\documentclass[oneside]{wissdoc}
% ----------------------------------------------------------------
% Diplomarbeit - Hauptdokument
% ----------------------------------------------------------------
% wissdoc Optionen: draft, relaxed, pdf, oneside --> siehe wissdoc.cls
% ------------------------------------------------------------------


% Packages für Deckblatt
\usepackage[absolute]{textpos} 	%Textboxen an absolute Position setzen
\usepackage{setspace}						%Zeilenabstand anpassen
\usepackage{color}							%Farbige Schrift
\usepackage{graphicx}						%Einbinden von Grafiken

% Weitere packages: (Dokumentation dazu durch "latex <package>.dtx")
% \usepackage{varioref}
% \usepackage{verbatim}
% \usepackage{float}    %z.B. \floatstyle{ruled}\restylefloat{figure}
%\usepackage{subfig}
\usepackage{wrapfig}
\usepackage{subfigure}
\usepackage[ngerman]{babel}
\usepackage[T1]{fontenc}
\usepackage[ansinew]{inputenc}

% Refernziere Kapiteln
\usepackage{hyperref}

% Zitaten
\usepackage[]{dirtytalk}
\usepackage{epigraph}

% Footnote
\usepackage{scrextend}



% Zeilenabstand nach Vorgabe - Falls gefordert
%\setstretch{1,3} 

% Inhaltsangabe auf Unterabschnitte(2 Ebenen) begrenzen
\setcounter{tocdepth}{2}


% \usepackage{color}    % Farbiger/grauer Text
% \usepackage{colortbl}   % Farbige/graue Tabellenzeilen und -spalten!! <--
% \usepackage{fancybox} % für schattierte,ovale Boxen etc.
% \usepackage{tabularx} % automatische Spaltenbreite
% \usepackage{supertab} % mehrseitige Tabellen
%% ---------------- end of usepackages -------------

%% Informationen für die PDF-Datei
\hypersetup{pdfauthor={Max Mustermann},%
            pdftitle={Bachelorarbeit},%
            pdfsubject={Titel der Arbeit},%
            pdfkeywords={Forschung, Entwicklung, Funktechnik},%
            pdfproducer={LaTeX},%
            pdfcreator={pdfLaTeX}
}

% Macros, nicht unbedingt notwendig
%%%%%%%%%%%%%%%%%%%%%%%%%%%%%%%%%%%%%%%%%%%%%%%%%%%%%%%%%%
% macros.tex -- einige mehr oder weniger nuetzliche Makros
%%%%%%%%%%%%%%%%%%%%%%%%%%%%%%%%%%%%%%%%%%%%%%%%%%%%%%%%%%


%%%%%%%%%%%%%%%%%%%%%%%
% Kommentare 
%%%%%%%%%%%%%%%%%%%%%%%
\ifnotdraftelse{
\newcommand{\Kommentar}[1]{}
}{\newcommand{\Kommentar}[1]{{\em #1}}}
% Alles innerhalb von \Hide{} oder \ignore{} 
% wird von LaTeX komplett ignoriert (wie ein Kommentar)
\newcommand{\Hide}[1]{}
\let\ignore\Hide

%%%%%%%%%%%%%%%%%%%%%%%%%
% Leere Seite ohne Seitennummer, wird aber gezaehlt
%%%%%%%%%%%%%%%%%%%%%%%%%

\newcommand{\leereseite}{% Leerseite ohne Seitennummer, n�chste Seite rechts (wenn 2-seitig)
 \clearpage{\pagestyle{empty}\cleardoublepage}
}

%%%%%%%%%%%%%%%%%%%%%%%%%%
% Neue Seite rechts, leere linke Seite ohne Headings
%%%%%%%%%%%%%%%%%%%%%%%%%%
\newcommand{\xcleardoublepage}
{{\pagestyle{empty}\cleardoublepage}}

%%%%%%%%%%%%%%%%%%%%%%%%%%
% Tabellenspaltentypen (benoetigt colortbl)
%%%%%%%%%%%%%%%%%%%%%%%%%%
\newcommand{\PBS}[1]{\let\temp=\\#1\let\\=\temp}
\newcolumntype{y}{>{\PBS{\raggedright\hspace{0pt}}}p{1.35cm}}
\newcolumntype{z}{>{\PBS{\raggedright\hspace{0pt}}}p{2.5cm}}
\newcolumntype{q}{>{\PBS{\raggedright\hspace{0pt}}}p{6.5cm}}
\newcolumntype{g}{>{\columncolor[gray]{0.8}}c} % Grau
\newcolumntype{G}{>{\columncolor[gray]{0.9}}c} % helleres Grau

%%%%%%%%%%%%%%%%%%%%%%%%%%
% Anf�hrungszeichen oben und unten
%%%%%%%%%%%%%%%%%%%%%%%%%%
\newcommand{\anf}[1]{"`{#1}"'}

%%%%%%%%%%%%%%%%%%%%%%%%%%
% Tiefstellen von Text
%%%%%%%%%%%%%%%%%%%%%%%%%%
% S\tl{0} setzt die 0 unter das S (ohne Mathemodus!)
% zum Hochstellen gibt es uebrigens \textsuperscript
\makeatletter
\DeclareRobustCommand*\textlowerscript[1]{%
  \@textlowerscript{\selectfont#1}}
\def\@textlowerscript#1{%
  {\m@th\ensuremath{_{\mbox{\fontsize\sf@size\z@#1}}}}}
\let\tl\textlowerscript
\let\ts\textsuperscript
\makeatother

%%%%%%%%%%%%%%%%%%%%%%%%%%
% Gau�-Klammern
%%%%%%%%%%%%%%%%%%%%%%%%%%
\newcommand{\ceil}[1]{\lceil{#1}\rceil}
\newcommand{\floor}[1]{\lfloor{#1}\rfloor}

%%%%%%%%%%%%%%%%%%%%%%%%%%
% Average Operator (analog zu min, max)
%%%%%%%%%%%%%%%%%%%%%%%%%%
\def\avg{\mathop{\mathgroup\symoperators avg}}

%%%%%%%%%%%%%%%%%%%%%%%%%%
% Wortabk�rzungen
%%%%%%%%%%%%%%%%%%%%%%%%%%
\def\zB{z.\,B.\ }
\def\dh{d.\,h.\ }
\def\ua{u.\,a.\ }
\def\su{s.\,u.\ }
\newcommand{\bzw}{bzw.\ }

%%%%%%%%%%%%%%%%%%%%%%%%%%%%%%%%%%%
% Einbinden von Graphiken
%%%%%%%%%%%%%%%%%%%%%%%%%%%%%%%%%%%
% global scaling factor
\def\gsf{0.9}
%% Graphik, 
%% 3 Argumente: Datei, Label, Unterschrift
\newcommand{\Abbildung}[3]{%
\begin{figure}[tbh] %
\centerline{\scalebox{\gsf}{\includegraphics*{#1}}} %
\caption{#3} %
\label{#2} %
\end{figure} %
}
\let\Abb\Abbildung
%% Abbps
%% Graphik, skaliert, Angabe der Position
%% 5 Argumente: Position, Breite (0 bis 1.0), Datei, Label, Unterschrift
\newcommand{\Abbildungps}[5]{%
\begin{figure}[#1]%
\begin{center}
\scalebox{\gsf}{\includegraphics*[width=#2\textwidth]{#3}}%
\caption{#5}%
\label{#4}%
\end{center}
\end{figure}%
}
\let\Abbps\Abbildungps
%% Graphik, Angabe der Position, frei w�hlbares Argument f�r includegraphics
%% 5 Argumente: Position, Optionen, Datei, Label, Unterschrift
\newcommand{\Abbildungpf}[5]{%
\begin{figure}[#1]%
\begin{center}
\scalebox{\gsf}{\includegraphics*[#2]{#3}}%
\caption{#5}%
\label{#4}%
\end{center}
\end{figure}%
}
\let\Abbpf\Abbildungpf

%%
% Anmerkung: \resizebox{x}{y}{box} skaliert die box auf Breite x und H�he y,
%            ist x oder y ein !, dann wird das uspr�ngliche 
%            Seitenverh�ltnis beibehalten.
%            \rescalebox funktioniert �hnlich, nur das dort ein Faktor
%            statt einer Dimension angegeben wird.
%%
% \Abbps{Position}{Breite in Bruchteilen der Textbreite}{Dateiname}{Label}{Bildunterschrift}
%

\newcommand{\refAbb}[1]{%
s.~Abbildung \ref{#1}}

%%%%%%%%%%%%%%%%%%%%
%% end of macros.tex
%%%%%%%%%%%%%%%%%%%%

% Print URLs not in Typewriter Font
\def\UrlFont{\rm}

\newcommand{\blankpage}{% Leerseite ohne Seitennummer, nächste Seite rechts
 \clearpage{\pagestyle{empty}\cleardoublepage}
}

%% Einstellungen für das gesamte Dokument

% Trennhilfen
% Wichtig!
% Im german-paket sind zusätzlich folgende Trennhinweise enthalten:
% "- = zusätzliche Trennstelle
% "| = Vermeidung von Ligaturen und mögliche Trennung (bsp: Schaf"|fell)
% "~ = Bindestrich an dem keine Trennung erlaubt ist (bsp: bergauf und "~ab)
% "= = Bindestrich bei dem Worte vor und dahinter getrennt werden dürfen
% "" = Trennstelle ohne Erzeugung eines Trennstrichs (bsp: und/""oder)

% Trennhinweise fuer Woerter hier beschreiben
\hyphenation{
% Pro-to-koll-in-stan-zen
% Ma-na-ge-ment  Netz-werk-ele-men-ten
% Netz-werk Netz-werk-re-ser-vie-rung
% Netz-werk-adap-ter Fein-ju-stier-ung
% Da-ten-strom-spe-zi-fi-ka-tion Pa-ket-rumpf
% Kon-troll-in-stanz
}

%Tabellen Kommandos
\newcolumntype{L}[1]{>{\raggedright\arraybackslash}p{#1}}
\newcolumntype{C}[1]{>{\centering\arraybackslash}p{#1}}
\newcolumntype{R}[1]{>{\raggedleft\arraybackslash}p{#1}}

% Index-Datei öffnen
\ifnotdraft{\makeindex}
%%%%%%%%%%%%%% includeonly %%%%%%%%%%%%%%%%%%%
% Es werden nur die Teile eingebunden, die hier aufgefuehrt sind!
%\includeonly{%
%titelseite,%
%erklaerung,%
%kurzfassung,%
%einleitung,%
%analyse,%
%entwurf,%
%implemen,%
%zusammenf%
%}
%%%%%%%%%%%%%%%%%%%%%%%%%%%%%%%%%%%%%%%%%%%%%%
\begin{document}
%Auskommentiert, da nicht notwendig für das Praktikum
%\ifnotdraft{
	%%%%Vorlage
	%%% Deckblatt - Hochschule Augsburg
%%%Deckblatt

\textblockorigin{20mm}{30mm}

\thispagestyle{empty}\null
%%%%Logo - Hochschule Augsburg - Informatik
\begin{textblock}{10}(8.0,1.1)
\begin{figure}[h]
	\centering
		\includegraphics[width=0.45\textwidth]{logos/hsa_informatik_logo_lq.pdf}
\end{figure}

\end{textblock}

%%% Text unter Logo
\begin{textblock}{15}(12.43,2.1)
	\LARGE
	\textsf{
		\textbf{\textcolor[rgb]{1,0.41,0.13}{\\
			\begin{flushleft}
				Faculty of\\
				Informatics\\
			\end{flushleft}
			}
		}
	}
\end{textblock}

%%%%Textbox links - Informationen
\begin{textblock}{15}(2,1.4)
	%\LARGE
	\begin{flushleft}
		\begin{spacing} {1.2}
			\huge	
				\textbf{Bachelor Thesis}
				\vspace{30pt}
				\\
				%\textcolor[rgb]{1,0.41,0.13}{\\
				%\textbf{Bachelorarbit}}\\
				\vspace{60pt}
			\LARGE
				Field of study\\
				Technische Informatik\\
				\vspace{30pt}
				Florian P�rvu\\
				\vspace{60pt}		
				Thesis Title:	C++ Crossplatform Bibliothek zur Performanceanalyse\\ \hspace{31mm}von Anwedungen unter dynamisch generierten Lasten\\
				\vspace{60pt}		
			\LARGE
				First examiner: Prof. Dr. Thomas Kirchmeier\\
				Second examiner: Prof. Dr. Hubert H�gl\\
				\vspace{10pt}		
				Submission date: 20.01.2022\\
			\end{spacing}
		\end{flushleft}
		
\end{textblock}



%%%%Textbox rechts - Hochschule
\begin{textblock}{5}(12.45,9.0)
	\scriptsize
	\textcolor[rgb]{1,0,0}{\\
		\begin{flushleft}
			\begin{spacing} {1.3}
				Hochschule f\"ur angewandte\\
				Wissenschaften Augsburg\\
				\vspace{4pt}
				An der Hochschule 1\\
				D-86161 Augsburg\\
				\vspace{4pt}
				Tel\hspace{2pt} +49 821 55 86-0\\
				Fax +49 821 55 86-3222\\
				www.hs-augsburg.de\\
				info(at)hs-augsburg-de
			\end{spacing}
		\end{flushleft}
		}
\end{textblock}


%%%%Textbox rechts unten - Fakult�t und Autor
\begin{textblock}{5}(12.45,11.5)
	\scriptsize
		\begin{flushleft}
			\begin{spacing} {1.3}
				Faculty of Informatics\\
				Tel \hspace{12pt} +49 821 55 86-3450\\
				Fax \hspace{10pt} +49 821 55 86-3499\\
				\vspace{6pt}
				Bachelor Thesis author\\
				Florian P�rvu\\
				Asternweg 2\\
				86399 Bobingen\\
				Tel +49 176 3693 7974\\
				pflorian306@gmail.com\underline{}\\
			\end{spacing}
		\end{flushleft}
	\end{textblock}
\pagebreak  %<-- Nach Vorgabe der HS Augsburg
	%
	%%%% Innere Titelseite 
 	%\include{titelseite} %<-- Vorgabe Prüfer oder frei wählbar
	%
	%%%%Optional - Falls von der Firma gefordert
	%\include{sperrvermerk}
	%
	%%%%Pflicht
 	%\include{erklaerung}
	%
	%%% Leere Seite bei zweiseitigem Druck
	%\ifnotonesideelse{\blankpage}{}
	%\include{kurzfassung}
	%%% Leere Seite bei zweiseitigem Druck
	%\ifnotonesideelse{\blankpage}{}
%}



%
%% ++++++++++++++++++++++++++++++++++++++++++
%% Verzeichnisse
%% ++++++++++++++++++++++++++++++++++++++++++
\pagenumbering{roman}
\ifnotdraft{
\tableofcontents
% Leere Seite bei zweiseitigem Druck
\ifnotonesideelse{\blankpage}{}
%\listoffigures
%% Leere Seite bei zweiseitigem Druck
%\ifnotonesideelse{\blankpage}{}
%\listoftables
%% Leere Seite bei zweiseitigem Druck
%\ifnotonesideelse{\blankpage}{}
}
%% ++++++++++++++++++++++++++++++++++++++++++
%% Hauptteil
%% ++++++++++++++++++++++++++++++++++++++++++
\graphicspath{{figures/}}
\pagenumbering{arabic}

%%% Ab hier eigene Kapitel einfügen
%%% Kapitel sind analog zur Wordvorlage zu wählen
%Einleitung.tex
%
\chapter{Introduction}
\section{Testing}
Software testing refers in the IT-Branch to a product with the main goal of finding bugs in a software program or application. These bugs refer to errors or faults and can occur because of a bad commands sequence specified in the program's source code.\\
A successful test can be achieved when its main requirements are met. Some examples would be the execution on different environments, time constraints or delivering expected outputs for randomly chosen inputs. Conditions are set by the tester and may vary depending on the use case.\\
\section{Motivation}
Most companies put a lot of effort in delivering high performance products to their customers. In order to do so, each of them test their gadgets, machines or software for possible failure scenarios. Now-a-days tests are being fully automated, which decreases the failure possibility that can happen because of human errors.\cite{8389562}\\ Unfortunately the creation of fully automated tests is not as easy as it may sound. Many testing developers know the struggle of finding the right tools for the job and by the end of testing phase, their project is filled with unnecessary dependencies that will overload the program and occupy valuable memory.
\begin{center}
	\say{As ironic as it seems, the challenge of a tester is to test as little as possible. Test less, but test smarter}
	\begin{flushright}
		\textit{Federico Teldo, Co-Founder Abstracta US}
	\end{flushright}
\end{center}
Additionally the problem enhances when a company reaches a certain size with a significant number of customers, which tend to run the product on different machines and architectures. In order to keep their customers, producers need to adapt their products to support newer or older machines. This makes software analysis even more difficult because of the increasing complexity, which comes with different systems. For this reason many companies dedicated themselves to creating programs that focus only on software examination and fixing. In time a new trend has been created with demands so high that rapidly developed itself into a new market section.\\
\section{Growing markets}
For the last decade the software testing market has been developing and now it has grown so big that it would be foolish to ignore it. According to \dq Global Market Insights\dq{} the testing software market has grown up to 40 billion dollars by the end of 2020 and is predicted to grow up to 60 billion dollars in the next 6 years.\cite{GMI}\\
\textit{\dq People may lie, but number don't\dq}. Multiple scientific papers and studies enforce this statement with different statistics of industries, which started adapting and reacting to this trend using the model \dq EaaS \dq{} which stands for \dq Everything as a service\dq{} and created \dq Testing as a Service\dq{}(TaaS). With this model customers not only pay for the current state of a product, but also for a service subscription, where they get updates and new features for the specific product and additional support from the company. The advantages that benefit the customer are set by the company for each of their subscription. (The basic rule is that you get more accurate results if you pay more). This service usually targets three groups: Developers, End Users and Certification Services.\cite{10.1145/1807128.1807153} 
\begin{figure}[htbp]
	\centering
	\includegraphics[width=0.80\textwidth]{../figures/einleitung/US_marketsize.png}
	\caption{US Software Market Size\cite{GMI}}
	\label{US_marketsize_graph}
\end{figure}

\subsection{Cost comparison}
People often tend to overlook costs of testing tasks that are insignificant in comparison to the total price of the project. However if these costs are reoccuring, their total price can go up to 40\% of the total project's costs\cite{8822082}. That amount of money covers not only for direct testing, which includes the staff, system and program testing, resources, computer time, etc. , but also for indirect testing, which refers to actions that take place because of poor direct testing, like rewriting code, additional analysis meetings or debugging.\\
It was proven that the cost of finding an error is about \$50 on average \cite{10.1145/1010773.1010774}, and is said that fixing an error after the software was released is four times more expensive than compared to if it was found during the testing phase.\cite{10.1007/978-981-10-8848-3_46} To avoid these financial expenses, companies train their developers to consider failure scenarios of the product early in the development stage.

\subsection{Common approaches}
In order to deliver a defect-free product, managers create a testing strategy.
To develop the most suitable strategy, they must identify the key components for it . This can be identified mostly by answering the following questions adapted from \cite{10.1007/978-981-10-8848-3_46}:
\begin{enumerate}
	\item Is the objective clear specified?
	\item What tools will be used?
	\item Is the system fully/partially automated?
	\item How will the test benefit the project?
\end{enumerate}
These questions should always be asked when a new feature that needs testing is being developed.
\section{Current Situation} 
At the time of writing this (December 2021), there are four mostly used architectures in the IT-Branch. These are Windows, Linux and MacOS. Linux was developed with the UNIX system as its core, while MacOS is only based on UNIX, which means that these are similar but not entirely compatible with each other. The big difference comes with Windows.\\
Each operating system can deliver informations about its own computer.

\footnotetext{Some companies also use ARM(aarch64), but this is not an OS and per default doesn't come with any process control system. In order to do so one needs to write or use an extra library for that solely purpose}

For that each of them has its own unique program. Windows uses the \dq Task-Manger\dq{} which comes with a GUI and shows real-time information about the CPU and memory of the computer, it comes with a list of processes and makes it possible for the user to manage them with only a couple of clicks.\\
Linux on the other hand is more text oriented. This operating systems comes with a built in command called "top". This also delivers informations about your system, but it doesn't allow you to manipulate processes like \dq Task-manger\dq{} does. Although not very practicable for developers, these tools allow the user to measure performance in a normal state where your computer is not put under pressure, but the most interesting state is when the CPU needs to do a lot of operations and it needs to share its valuable time with other processes. To force such situations, the user can use online stress tests, which use the browser as an workload source, which for many is not very practicable, because it uses an additional program, which runs in the background. Online stress tests rely on internet to work and a stable ethernet or wifi connection to work. This way the results can be very inaccurate if the the network is under pressure.
\begin{figure}[h]
	\centering
	\includegraphics[scale=0.25]{figures/einleitung/browser_comparisson.png}
	\caption{Browser CPU utilization comparison\cite{6724273}}
	\label{browser_comp}
\end{figure}
\\
There are already some alternative libraries on the internet to recreate this method, but most of them are written in other programming languages. Adding them to a C++ Project would rise the complexity of the source code.\\
Some people would think "Well C++ is a new and modern language. There must be something similar out there". This statement is true and because of its young age there are a limited number of officially tested methods. You could also take a look at the headers implemented in the \href{https://en.cppreference.com/w/cpp/header}{C++ standard library} \cite{CppRef} to familiarize yourself with common algorithms and data structures.

\section{Obective}
The purpose of this thesis is the development, implementation and testing of a Library for C++ applications that creates an artificial workload and delivers system independent statistics based on multi-threading. This will allow developers to implement their own stress tests and run simulations with different workloads, process priorities and scheduling, which can be used in different manners to achieve real-time applications and create benchmarks for each new product. These could be later automatized by using programs like \href{https://www.jenkins.io/}{Jenkins}\cite{Jenkins} or \href{https://about.gitlab.com/}{Gitlab}\cite{GitLab}.

\section{Implementation}
The library has two main components. The first one is the workload, which can vary depending of the user's input. This will create a system specific amount of threads and make them run a simulation function. The number of threads will stay constant and the time for running the workload task will be set accordingly for each input.\\
The second components is the system. This comes with functions, which can easily change a thread's priority, a system's scheduling policy and deliver statistics of the user's computer.\\
Basic operations are implemented to work on most operating systems, but there are some exceptions because of the differences between OS implementations, which makes them independent from one another.
\section{Advantages and Disadvantages}
\subsection{Pros}
\begin{enumerate}
	\item No additional expenses for third party software or subscriptions
	\item Can be directly added to the source code, which makes testing easy
	\item No external influence from the browser(ads) or internet connection, allowing efficient testing even offline
	\item Allows the creation of tests based on user experience 
	\item Companies don't have to give their data 
\end{enumerate}

\subsection{Cons}
\begin{enumerate}
	\item Tests don't come already prepared
	\item Developers have to add the source code to their build system or add the build files(CMakeLists.txt) to their own (if they already use CMake)
\end{enumerate}
\section{Summary}
With this library a company can save money, improve their product based on user feedback and create an internet independent test system for very minimal effort.  % Einleitung
\chapter{State of the art}
This chapter summarizes already existing technologies and implementations that are relevant to the library. Despite the fact that most of them are written in the C language, they are still relevant as they describe how many operating systems work and where many C++ standards come from.   
\section{Current Situation} 
At the time of writing this (December 2021), there are four mostly used architectures in the IT-Branch. These are Windows, Linux and MacOS. Linux was developed with the UNIX system as its core, while MacOS is only based on UNIX, which means that these are similar but not entirely compatible with each other. The big difference comes with Windows.\\
Each operating system can deliver informations about its own computer.

\footnotetext{Some companies also use ARM(aarch64), but this is not an OS and per default doesn't come with any process control system. In order to do so one needs to write or use an extra library for that solely purpose}

For that each of them has its own unique program. Windows uses the \dq Task-Manger\dq{} which comes with a GUI and shows real-time information about the CPU and memory of the computer, it comes with a list of processes and makes it possible for the user to manage them with only a couple of clicks.\\
Linux on the other hand is more text oriented. This operating systems comes with a built in command called "top". This also delivers informations about your system, but it doesn't allow you to manipulate processes like \dq Task-manger\dq{} does. Although not very practicable for developers, these tools allow the user to measure performance in a normal state where your computer is not put under pressure, but the most interesting state is when the CPU needs to do a lot of operations and it needs to share its valuable time with other processes. To force such situations, the user can use online stress tests, which use the browser as an workload source, which for many is not very practicable, because it uses an additional program, which runs in the background. Online stress tests rely on internet to work and a stable ethernet or wifi connection to work. This way the results can be very inaccurate if the the network is under pressure.
\begin{figure}[h]
	\centering
	\includegraphics[scale=0.25]{figures/einleitung/browser_comparisson.png}
	\caption{Browser CPU utilization comparison\cite{6724273}}
	\label{browser_comp}
\end{figure}
\\
There are already some alternative libraries on the internet to recreate this method, but most of them are written in other programming languages. Adding them to a C++ Project would rise the complexity of the source code.\\
Some people would think "Well C++ is a new and modern language. There must be something similar out there". This statement is true and because of its young age there are a limited number of officially tested methods. You could also take a look at the headers implemented in the \href{https://en.cppreference.com/w/cpp/header}{C++ standard library} \cite{CppRef} to familiarize yourself with common algorithms and data structures.
Not all operating systems will be covered in this thesis.
The following statements are true for the testing machines used in chapter 4?.
\section{Processes}
To many operating systems a process is like an wrapper defined by the kernel in order to allocate resources to an executing program.\\
When a process is created the system assigns him a \textit{process unique identifier} also called PID(a positive integer). Each process has its own PID, so two different executing programs cannot have the same PID \footnote{On UNIX like machines the first process to be called is the init process with PID 1}. The methods used to create a new process differ on each operating systems.
On Windows you can create processes by calling the \texttt{CreateProcessA()} function \cite{createProcWinAPI}, which returns a HANDLE (the equivalent PID for windows systems) and on linux you can use \texttt{fork()}\cite{LPI}. Each operating system calls one these methods internally every time the computer or the user starts a routine of execution, like starting a service, or a program (browser, spotify, etc). You can think of this system like a binary tree with one or more children with the kernel on top. The children have also an additional attribute called PPID, that contains the PID of its creator. If this is specified to zero then the kernel is the parent.\cite{wikiPPID}\footnote{There a very few processes that have the PPID=0 (ex. init)}.
\subsection{Threads} 
Each process has at least one thread of execution called the main thread, which as the name say contains the \texttt{main()} function. Once a thread has been created, the main thread has to wait for it to finish by calling a method called \texttt{join()}. As their creators, threads also have unique identifiers called \textit{Thread identifiers} or TID and OS-specific methods to create them. On UNIX systems one would use the \texttt{pthread\_create()} and on windows \texttt{Create\_Thread()}. But the role of this library is to make this whole process as easy as possible, so we will use the C++ Standard Library's threads (std::thread) and \dq detach\dq{}\footnote{There is also a method called \texttt{detach()} which allows a thread to separates itself from the main thread. This way the thread can terminate itself and the main thread doesn't have to wait for that thread's termination} ourselves from the other ones.\\
Processes have their own stack, so they can't communicate with each other. This is a huge problem when it comes concurrency (explained in \autoref{ssec:c	oncurrency}), but threads share the stack of their creator-process.
\subsection{Differences between threads and processes} 
Many people tend to think of threads and processes as being the same, but they are quite different. First as mentioned above threads share the same stack of the creator-process, while processes need intercommunication tools like pipes to talk to each other. Another difference is that a process can have multiple threads attached to it, but a thread cannot belong to more than one process.\footnote{If the execution method of a thread calls \texttt{fork()} then that process's PPID will be the PID of thread's creator}. Their identifiers are also independent from their creator's. This way a TID can be equal to its creator's (or another process's) PID. \\
One can imagine a process like an octopus and the threads being it's arms. One octopus has many arms that can execute multiple tasks at the same time, but an arm cannot belong more than one octopus.\\
In this library we use multiple threads to simulate a user specific workload because this way we can time their execution start point with only one shared variable and we don't have to worry about interprocess communications. 
\subsection{Attributes}
Normally when we would create a threads using the unix methods, we would pass a pointer to a
structure that describes the attributes for that specific thread(that structure can be created with 
\texttt{pthread\_attr\_init()} and be destroyed with \texttt{pthread\_attr\_destroy()}. Some of these attributes include the 
scheduling priority, scheduling policy and stack size, which are important for out tests. 
Unfortunately the standard library doesn't have this option. In order to set and get this attributes
we need to use the architecture's dependent functions. 
\subsection{Stack Size}
\begin{wrapfigure}{1}{0.5\textwidth}
	\centering
	\includegraphics[width=.98\linewidth]{../figures/systemgrundlagen/stack.png}
	\caption{Stack segment}
	\cite{stack}
\end{wrapfigure}
The stack is a piece of memory where meta-data and local variables are saved when the \texttt{main()} function calls a routine/method. This memory segment is limited and doesn't allow an infinite number of data segments (also called stack frames) being stored in it. The stack uses assembly instruction like \texttt{pop} to delete a frame and \texttt{push} to add a frame. For consistency the stack will always pop the last element pushed. This is also known as \dq Last in First Out\dq{}\cite{stack}.\\
On UNIX systems we would create an attribute structure and pass the
desired options there. But because we don't use the unix's system function \texttt{pthread\_create()} we also
cannot use the attribute structure. Furthermore the standard library doesn't support such tweaks. 
This is very important if someone wants to use this for an ARM architecture, because he won't be
able to define a meaningful stack size. This doesn't pose any threats for many operating systems out
there, but for ARM, which has a limited stack size can be problematic. For windows this attribute
can be set using the \texttt{Create\_Thread()} function just like in unix using \texttt{pthread\_create()}.
\newpage
\section{Concurrency}
\label{ssec:concurrency}
Concurrency means that two or more things are happening simultaneously. This pehnomenon is happening
everyday almost everywhere we look. Even we as humans are capable of such thing, for example walking
and talking at the same time. In computer science concurrency means that more than one process can
be executed at the same time. Many systems have this ability, because most of them are
multiprocessor computers. Nowadays many systems measure the concurrency of a system by its number of hardware threads. This
unit of measure tells us how many independent tasks can the processor handle. Even some single core computers can handle concurrency to some extent.
One calculation unit can handle one task at a time, but it can quickly switch to another task if
necessary. This is why sometime even single core units give the impression of resolving jobs
simultaneously\cite[Chapter~1]{concurrency}. Although performant this method does not come without its flaws. When a cpu gets a new task, the resources of the old task (eg. local variable, meta data of the current task, etc.)will be replaced by those of the new job. This is also known as \dq Context Switching\dq{}.\\
\subsubsection{Context switching}
Context switching can be triggered with an interrupt or a syscall. One example would be when another thread/process with a higher priority starts. This happens because the system would have to execute \texttt{fork()} for processes or \texttt{clone()} for threads which are also system calls. This would trigger the cpu
to reschedule his task based on the schedule and priority of the tasks in the tasks list\cite{clone}.
The reordering of tasks after switching CPUs will be explained in detail in \ref{priorities}.\\ 

\section{CPU Affinity}
\label{ssec:cpu_affinity}
The affinity of a CPU determines the number of processors that one process can use for its threads.
This library offers methods to restrict the number of CPUs off which a process can run on. This is
sometimes desirable because of the following reasons:
\begin{enumerate}
	\item Data invalidation: When a process starts, the user cannot tell from outside on which CPU
	that thread started. When a process finishes its time-slice, he has to give up its CPU for
	others to use it and it come back later, but it won't necessarily start on the same CPU as the
	last time. \footnote{This is a part of context switching, which was discussed in chapter \ref{ssec:concurrency}}. When this happens, entries in that CPU's cached must be replace or removed,
	which is known as \dq Cache invalidation\dq{}. This is not a flawless method and cache inconsistencies
	can appear.
	\item Emergency CPU: On real-time systems, where human lives are at risks, many developers will
	deliberately block some CPUs ( on a multicore machine) to use them when the system returns
	erros and needs to immediately execute safety protocols. Because the CPUs were blocked
	from being used on other processes, these remain free and so the execution of the safety procedure can start without any delay (eg. context switching).
	
\end{enumerate}

By default most systems allow each process to use all CPUs. If the user turns off half of the
CPUs of a given process and tries to create an additional workload with the library's methods for that process, he must keep in
mind that the workload will also be cut in half because the threads have less processors to work on.
\section{Priorities}
\label{priorities}
When it comes to the priority of a process there is a big difference between a UNIX system and a
windows machine. On Windows the priority is determined by the priority class of the process and the
its thread priority.\\
\subsection{Windows}
\label{winPrioClass}
Based on the winAPI documentation\cite{priorityClasses}, the classes can have the following values:
\begin{enumerate}
	\item IDLE\_PRIORITY\_CLASS (0x00000040): This is the lowest priority, processes belonging to this class run only if the system is idle and can be preempted\footnote{If a process is preempted that means it stops executing and yields the cpu} by a process with a higher priority
	\item BELOW\_NORMAL\_PRIORITY\_CLASS(0x00004000): This class has a higher priority than an idle-classed process but a lower priority than a normal-classed process 
	\item NORMAL\_PRIORITY\_CLASS(0x00000080
	): This is the default class for all processes created by the user
	\item ABOVE\_NORMAL\_PRIORITY\_CLASS(0x00008000): This class has a higher priority than an normal-classed process but a lower priority than a high-classed process
	\item HIGH\_PRIORITY\_CLASS(0x00000080): This class is usually used for time critical jobs
	\item REALTIME\_PRIORITY\_CLASS(0x00000100): This is the class with the highest priority and is rarely used because it stop most of the tasks on the calling machine
\end{enumerate}
Each class can be preempted by a higher priority class besides the realtime-class. Classes categorize only processes, but not their created threads. For these the following values can be set:
\begin{enumerate}
	\item THREAD\_PRIORITY\_IDLE(-15)
	\item THREAD\_PRIORITY\_LOWEST(-2)
	\item THREAD\_PRIORITY\_BELOW\_NORMAL(-1)
	\item THREAD\_PRIORITY\_NORMAL(0)
	\item THREAD\_PRIORITY\_ABOVE\_NORMAL(1)
	\item THREAD\_PRIORITY\_HIGHEST(2)
	\item THREAD\_PRIORITY\_TIME\_CRITICAL(15)
\end{enumerate}
These are similar to the classes mentioned above and can be interpreted alike.\\
\subsection{Linux}
On Linux however, the priority of a process is harder to be determined. This value is composed out of two main components: the nice value of the process and its thread priority.
\subsubsection{Nice Values}
Nice values can range from 20 to
-19 with 20 being the nicest value and so the smallest priority and -19 being the worst value and so
the highest priority. For a better understanding one could think that a process is nice, when he doesn't need
the CPU and so it lets other threads to use it. In my research one thing was mentioned and that is a low nice value (hence a high priority) doesn't mean other processes won't get any CPU
time. The scheduler will make them more favorable, but other processes will also get their turn for the
CPU.
To change the nice value of a process, in this library, I am using
the calls \texttt{getpriority()} and \texttt{setpriority()} from the header \texttt{sys/resource.h}.\\
There is one
critical thing that the caller needs to know. In order to increase the nice value of the calling
process, the user can use the given methods of the library and additionally use the command
\texttt{sudo setcap cap\_sys\_nice=ep PATH/TO/EXECUTABLE} on the built binary (the command \texttt{setcap} can change the executable to
run as a privileged process, but only when called as root or with the keyword sudo), run it as root or build the executable
program as the root-user from the beginning. You need to do this extra step,
because by default any user-created processes are unprivileged.\\
Unprivileged processes can lower their own
priority, but are not allowed to increase it more than the value of the operation
\textit{20-RLIMIT\_NICE}. The RLIMIT\_NICE is resource on your UNIX machine and can be set/gotten with the methods \texttt{getrlimit()} and \texttt{setrlimit()} respectively. These functions takes as arguments an integer, which describes the resource we want to get or set (in this case RLIMIT\_NICE) and a \texttt{struct rlimit} pointer, which describes the priority of the given resource. The structure \texttt{rlimit} has two attributes: the \dq rlim\_cur\dq{}(also called the soft limit), that represent the current value of the process and the \dq rlim\_max\dq{}(also called the hard limit or ceiling), which tells one user the limit of which that process can be set to. On my testing system
RLIMIT\_NICE is set to 13 and the limits for processes compiled by my users are zero.
\begin{figure*}[!htb]
	\centering
	\subfigure[RLIMIT\_NICE code]{
		\label{RLIMIT_NICE_code}
		\includegraphics[width=0.9\textwidth]{../figures/systemgrundlagen/RLIMIT_NICE_code.png}}
	\subfigure[RLIMIT\_NICE output]{
		\label{RLIMIT_NICE_output}
		\includegraphics[width=0.9\textwidth]{../figures/systemgrundlagen/RLIMIT_NICE_output.png}}
	\caption{RLIMIT\_NICE} 
	\label{RLIMIT_NICE}
\end{figure*}
\\
Per default the process will have
the nice value of 0 and this value can be increased to the highest value allowed (19), but cannot be
decreased afterwards to a value lower than \textit{user\_nice\_value = 20-RLIMIT\_NICE}. A way of
increasing the nice value would be to increase the RLIMIT\_NICE value (also as root or with root-rights = sudo), which will allow a normal user to increase the value given until it reaches
\dq user\_nice\_value\dq{} or log in as root, use the library's functions, build the program and set the SUID as root.
\footnote{The SUID is a special bit that one can set and allows normal users to run the program as they
were root}\\
At last you could also modify the \dq/etc/security/limits.conf\dq{} file and set a new max nice value for a
certain user, but this is not the best solution, because that user would have the power to change
priorities not only for one program, but for all programs.
\subsubsection{Scheduling}
\label{ssec:sched_policies}
Unlike Windows, Linux has methods to change one's process and its threads scheduling policies.The default policy set on UNIX
is called "Round-Robin Timesharing" (SCHED\_OTHER). This allows jobs to be executed in a round robin fashion where
each process gets an equal time-slice of a CPU. There are more than one policy which can be set.
These are:
\begin{enumerate}
	\item SCHED\_OTHER
	\item SCHED\_BATCH
	\item SCHED\_IDLE
	\item SCHED\_FIFO
	\item SCHED\_RR
\end{enumerate}
The difference between SCHED\_RR and "Round Robin Timeshare"(SCHED\_OTHER) is that the realtime policy lets
us to coordinate the priorities for that schedueling policy's queue.\\
The differences are that BATCH schedules a process less
frequently, if the it gets the CPU very often and IDLE is the equivalent to a process with a nice value of 19
(very nice process <=> lowest value).
SCHED\_BATCH and SCHED\_IDLE are two normal prioritised policies, which differ from SCHED\_OTHER, but
not enough for me to focus too much on them.\\
You can get the current policy of your process by calling \texttt{int sched\_getscheduler(pid\_t pid)} from the \texttt{<sched.h>} header file. 
\\
Each of the policies mentioned above has a range of priorities levels, which can be get using the \texttt{sched\_get\_priority\_max(int policy)} and \texttt{sched\_get\_priority\_min(int policy)} methods found in \texttt{<sched.h>}.\footnote{These can be different depending on the calling xUNIX
machine}\\
On my Linux Notebook I have the following values:
\begin{figure*}[!htb]
	\centering
	\subfigure[sched\_prio\_range code]{
		\label{sched_prio_range_code}
		\includegraphics[width=0.9\textwidth]{../figures/sched_prio/sched_prio_range_source.png}}
	\subfigure[sched\_prio\_range output]{
		\label{sched_prio_range_output}
		\includegraphics[width=0.9\textwidth]{../figures/sched_prio/sched_prio_range_output.png}}
	\caption{sched\_prio\_range} 
	\label{sched_prio_range}
\end{figure*}
You can only change the priority of real-time policies. hence when you set the policy of a thread to OTHER, IDLE or BATCH you can't change the priority of that process.\\
As you can observer only two of these have a \dq real\dq{} range.
SCHED\_RR and SCHED\_FIFO are characterized as real-time policies and have a higher priority than the
others. When two processes with SCHED\_RR and respectively SCHED\_FIFO have to share a CPU,
ironically the one that was placed first in that CPU's queue will get to run its job. Both of these
policies can lose access of their CPU, if they finish execution, yield\_sced() or a syscall is called
and a higher priority process preempts them (a process with a lower nice value appears in the queue
or the user changes the value himself). SCHED\_RR can also lose its access if the timeslice of
the job expires.

\section{Synchronization}
When working with threads, the programmer cannot forget that these share the same stack, therefore they share the same variables. When two or more threads start their execution (eg. a simple addition), one cannot tell when will they perform what (if their priorities weren't tampered with).\\
There are two main operations threads can perform on a variable: \texttt{read()} and \texttt{write()}. Reading from a variable is not problematic, because its content will always stay the same, but writing to one is a whole different story. Let's take a look at the following code:
\begin{figure}[!htb]
	\centering
	\includegraphics[width=0.5\textwidth]{../figures/systemgrundlagen/synchronizationExample.png}
	\caption{Thread's behavior without synchronization - Source code}
\end{figure}
Here we create two threads with the sole purpose of adding a common variable. Most people would expect, because of the order of creation, for t1 to do his job and afterwards t2, but this is not the case.
\newpage
%\begin{figure}[!h]
%	\centering
%	\includegraphics[width=0.5\textwidth]{../figures/systemgrundlagen/synchronizationExampleOutput.png}
%\end{figure}
\begin{wrapfigure}{1}{0.5\textwidth}
	\centering
	\includegraphics[width=.98\linewidth]{../figures/systemgrundlagen/synchronizationExampleOutput.png}
	\caption{Thread's behavior without synchronization - Output}
\end{wrapfigure}

As you can observe, these prints make no sense. The threads do not follow a sequentially pattern so the variable can be increased by \texttt{t1} for a time then by \texttt{t2} and for the rest of the remaining time by \texttt{t1} again.\\
Something that cannot be seen in the output would be the scenario when \texttt{t1} and \texttt{t2} try to access the variable and increment it at the same time. No one can accurately predict what the result would be (this is also known as \textit{ Unexpected behavior}). To resolve this problem in classic C people came up with the idea of a locking mechanism called \texttt{Mutex}.\\
\subsection{Mutex}
Mutexes are guards that can block other threads the access to variables inside a user-defined block of code (also known as \textit{scope}). These allow the thread that called \texttt{lock()} exclusive access to the variables inside that scope until the thread calls \texttt{unlock()}\cite{concurrency}. If another thread tries to lock the mutex for himself while the mutex is being used, then it will land in a waiting state until that lock is released.  The accessing order of the variable is \dq First come, first served\dq{}.\\
But as many implementations, this method can create problems, especially if the programmer forgets to \texttt{unlock()} the mutex. If this ever happens, it can lead to other threads being stucked in a waiting state forever (this is also called a \textit{Deadlock})\cite{cppExpert}.\\
This method requires a lot of concentration from the programmer and a good overview of all locking and release point.
Fortunately the \textit{Standard C++ Library} implemented new ways to use mutexes like \textit{lock\_guards}, which unlock themselves when the user goes out of scope, \textit{unique\_locks}, \textit{shared\_locks} and even \textit{timed\_mutexes}. Additionally they made a more simplistic technology for threads to access only common variable (and not a whole scope) called \textit{Atomic Variables}.
\subsection{Atomic Variables}
Atomic variables can be seen as wrappers for primitive data types, like \texttt{bool}, \texttt{int}, \texttt{double}, \texttt{float}, etc., but also for defined structs like \texttt{uint32\_t}, \texttt{uintptr\_t} or \texttt{int\_fast32\_t}\\
Basic syntax of this wrapper is: \textit{atomic<data\_type/class>} or (if defined) \textit{atomic\_data\_type} and it will grant the calling thread exclusive access to a certain variable while reading from or writing to it.\\
To read from an atomic variable one could just use the \dq =\dq{}-operator or (preferred) the \texttt{atomic\_var.load()} method. Writing to an atomic variable should only be done by using the \texttt{atomic\_var.store()} method, because this way you can expect that no one else besides the calling thread is accessing that variable\cite{stdAtomic, atomicConference}.\\
One must be very careful when using atomic operations on a variable.
To explain this in detail let's observe the following example:
\begin{center}
	A simple incrementation of a variable:
	\begin{enumerate}
		\centering
		\item x++
		\item x += 1
		\item x = x+1
	\end{enumerate}
\end{center}
All of these are atomic operation, but number three is different from the first two. The first and second operation are both \textit{atomic increments}. The third however, needs to apply more operations on the variable. Here there is an \textit{atomic read}, to get the value from x and store it in a register, afterwards we add one to it and there is an \textit{atomic write}, to set the variable to its new value. If you are working with multiple threads any other thread can come in between the \texttt{write} and the \texttt{read} and temper with the value, because there is no exclusivity while the value is stored in the register. 
The atomic library comes with additional member functions to make sequences of reading and writing easier and specialized member function that define bitwise operations on a variable, but they go way too deep in complexity and understanding for me to focus too much on them.
\section{CMake}
Despite the numerous implementations for the C++ language, there are still some hardships that need to be overcome when working with it. When it came to building, many people used os-dependent technologies such as the \textit{GNU Compile Collection}(short GCC) for Linux\cite{gcc} or MingW, which supports the usage of the GCC compiler, on Windows\cite{mingw}. Despite the existing methods, the demand \dq for a cross-platform build environment\dq{} was so high a new tool called \textit{CMake} emerged\cite{cmake-overview}.\\
\dq CMake is an extensible open-source system that manages the build process in an operating system and in compiler-independent manner\dq{}. As mentioned on their website, the system is used with native build environments and can be used to build binaries of a specific projects. These binaries can be executable files or compiled libraries such as \text{*.o}-files (also known as Object-files) on Linux or \textit{*.dll} (Dynamical Linked Library) on Windows. In order to use this program one has to first write basic text files and call them \textit{CMakeLists.txt}. The content of these files needs to be written in an interpreted language. Its Syntax goes as follows: \texttt{COMMAND(args...)}. Each subdirectory of the project needs to contain a \textit{CMakeLists.txt} file that will later be linked through a command to the main \textit{CMakeLists.txt} file. 

\section{Need for action}
Although the presented technology uses the C and C++ syntax and logic, there is still no universal tool to cover the most relevant operating systems out there when it comes to the machine's performance. This is where this library comes in handy.\\
The goals are:
\begin{enumerate}
	\item summarize the complexity between operating systems into easy-to-understand methods
	\item supply the user with numerous operations to ensure homogeneity and flexibility usage in order to cover most user-cases in the industry 
	\item  encourage creative test-solving ways   
\end{enumerate}
% Nachfragen :S es sieht nicht ganz richtig aus 
%warum ge0stae ich die Library so wie sie ist? %Systemgrundlagen
%% Beispiele.tex
%%

\chapter{Beispiele}
\label{ch:Beispiele}
%% ==============================
Bla fasel\ldots

Beispiele

\section{Zitieren}
Quellen\cite{li00,jackson91,lakhina04a,netflow,rfc2386} 
nicht vergessen. Dazu verwendet ihr bibtex.

%% ==============================
\section{Bild einf�gen}
%% ==============================

\subsection{Ein Bild skaliert}

\begin{figure}[htbp]%Positionierung vorzugsweise an dieser Stelle. Falls nicht m�glich oben positionieren. Falls das auch nicht geht unten.
	\centering
		\includegraphics[width=0.80\textwidth]{../figures/large1.png}
	\caption{Beschriftungstext}
	\label{fig:large1}
\end{figure}

\subsection{Zwei Bilder nebeneinander oder untereinander}
%%%%%%%%%%%%%%%%%%%%%%%%%%%%%%%
\begin{figure*}[!htb]
	\centering
	\subfigure[Beschriftung Bild links]{
	  \label{fig:small1}
		\includegraphics[angle=0,width=0.68\textwidth]{../figures/small1.png}}
	\subfigure[Beschriftung Bild rechts]{
	  \label{fig:small2}
		\includegraphics[angle=0,width=0.68\textwidth]{../figures/small2.png}}
 	\caption{Beschriftung beide Bilder} 
	\label{fig:beidebilder}
\end{figure*}
%%%%%%%%%%%%%%%%%%%%%%%%%%%%%%%


%% ==============================
\section{Tabellen}
%% ==============================
\begin{table}[htbp]
	\centering
		\begin{tabular}{|l|L{3.3 cm}|L{6.1 cm}|}
			\hline
			Firma								&			Produkte / L�sungen											&		WEB\\
			\hline
			Concentrix (Soitec)	&	Module mit Konzentratoren (Fresnel-Linsen)	&	http://www.soitec.com \\
			\hline
			Isofoton						&	Module mit Konzentratoren (Fresnel-Linsen)	&	http://www.isofoton.com \\
			\hline
			Semprius						& Module mit Konzentratoren (Fresnel-Linsen)	& http://www.semprius.com \\ 
			\hline
			\hline
			Azur Space					& Mehrfach Junction Zellenhersteller					& http://www.azurspace.com \\
			\hline
			Cyrium Technologies	& Mehrfach Junction Zellenhersteller					& \small{http://www.cyriumtechnologies.com} \\
			\hline
			Emcore							& Mehrfach Junction Zellenhersteller					& http://www.emcore.com \\
			\hline
		\end{tabular}
	\caption{Hersteller von CPV-Produkten}
	\label{tab:Hersteller}
\end{table}

\begin{table}[htb]
		\centering
		%\renewcommand{\arraystretch}{1.03}
		\caption{Single-hop Scenario - Traffic Pattern \label{t:traffic}}
			
		\begin{tabular}{l@{~}l@{\,\,}l@{\,\,}l} \hline \rule{-2pt}{12pt}
			Pattern& Parameter & Distribution & Range/Values  \rule{0pt}{12pt} \\ \hline \rule{-2pt}{12pt}  
      \textbf{Burst}      
      & Burst IAT         & uniform  & [9.9; 10.1] s\\ 
      & Packets per Burst & constant & 100\\
      & Packet IAT        & constant & 0.02 s\\
      & Packet Size       & constant & 1024 bit\\
      & \# Sources & -        & 2\\
			& Offset						& uniform  & [0; 1] s\\ 
      \hline      \hline\rule{-2pt}{12pt} 
      \textbf{Single}     & Packet IAT        & uniform  & [0.9; 1.1] s\\
      & Packet Size       & constant & 1024 bit\\
      & \# Sources & -        & [10;20;30;40;50;\\
      & & & 60;70;80;90;100]\\
			& Offset						& uniform  & [0; 1] s\\ 
      \hline
    \end{tabular}
\end{table}   % Beispiele
%% analyse.tex
%%

\chapter{Analyse}
\label{ch:Analyse}
%% ==============================
Bla fasel\ldots

%% ==============================
\section{Abschnitt 1}
%% ==============================
\label{ch:Analyse:sec:Abschnitt1}

Bla fasel\ldots

%% ==============================
\section{Abschnitt 2}
%% ==============================
\label{ch:Analyse:sec:Abschnitt2}
Bla fasel\ldots
\subsection{Unterabschnitt}
Bla fasel\ldots
\subsubsection{Unter-Unterabschnitt}

     % Analyse


%% ++++++++++++++++++++++++++++++++++++++++++
%% Anhang
%% ++++++++++++++++++++++++++++++++++++++++++

%\appendix
%\include{anhang_a}
%\include{anhang_b}

\ifnotonesideelse{\cleardoublepage}{}

%% ++++++++++++++++++++++++++++++++++++++++++
%% Literatur
%% ++++++++++++++++++++++++++++++++++++++++++
\addcontentsline{toc}{chapter}{\bibname}
%  mit dem Befehl \nocite werden auch nicht zitierte Referenzen abgedruckt 
% (normalerweise nicht erwünscht)
% \nocite{*}
\bibliographystyle{unsrt}
%Einbinden Bibtexdatei - Direkt aus JabRef generiert
\bibliography{literatur}
%\bibliography{links}
%% ++++++++++++++++++++++++++++++++++++++++++
%% Index (optional)
%% ++++++++++++++++++++++++++++++++++++++++++
%\ifnotdraft{
%\addcontentsline{toc}{chapter}{Index}
%\printindex            % Index, Stichwortverzeichnis
%}
\listoffigures
\end{document}
